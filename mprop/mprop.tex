\documentclass{mprop}
\usepackage{graphicx}
\usepackage{amssymb,amsmath,amsthm}
\usepackage{url}
%\usepackage{fancyvrb}
%\usepackage[final]{pdfpages}

\theoremstyle{definition}
\newtheorem{definition}{Definition}
\DeclareMathOperator*{\argmax}{arg\,max}

\begin{document}
\title{Variational Inference for Inverse Reinforcement Learning with Gaussian
  Processes}
\author{Paulius Dilkas}
\date{Date of submission placed here} % TODO
\maketitle
\tableofcontents
\newpage

\newcommand{\Kuu}{\mathbf{K}_{\mathbf{u},\mathbf{u}}}
\newcommand{\Krr}{\mathbf{K}_{\mathbf{r},\mathbf{r}}}
\newcommand{\Kru}{\mathbf{K}_{\mathbf{r},\mathbf{u}}}
\newcommand{\DKL}{D_{\mathrm{KL}}}
\newcommand{\pfull}{p(\mathcal{D}, \Theta, \mathbf{X_u}, \mathbf{u}, \mathbf{r})}
\newcommand{\posterior}{p(\Theta, \mathbf{u}, \mathbf{r} | \mathcal{D}, \mathbf{X_u})}
\newcommand{\approximation}{q(\Theta, \mathbf{u}, \mathbf{r}; \Lambda)}

\section{Introduction}
%briefly explain the context of the project problem

Inverse reinforcement learning (IRL)---a problem proposed by Russell in 1998
\cite{DBLP:conf/colt/Russell98}---asks us to find a reward function for a Markov
decision process that best explains a set of given demonstrations. IRL is
important because reward functions can be hard to define manually
\cite{DBLP:conf/icml/PieterN04,DBLP:journals/corr/abs-1806-06877}, and rewards
are not entirely specific to a given environment, allowing one to reuse the same
reward structure in previously unseen environments
\cite{DBLP:journals/corr/abs-1806-06877,DBLP:conf/uai/JinDAS17,DBLP:conf/nips/LevinePK11}.
Moreover, IRL has seen a wide array of applications in autonomous vehicle
control
\cite{DBLP:journals/ijsr/KimP16,DBLP:journals/ijrr/KretzschmarSSB16}
and learning to predict another agent's behaviour
\cite{DBLP:journals/ai/BogertD18,DBLP:conf/aaai/VogelRGR12,ziebart2008maximum,DBLP:conf/huc/ZiebartMDB08,DBLP:conf/iros/ZiebartRGMPBHDS09}.
Most approaches in the literature (see Section \ref{literature}) make a
convenient yet unjustified assumption that the reward function can be expressed
as a linear combination of features. One proven way to abandon this assumption
is by representing the reward function as a Gaussian process
\cite{DBLP:conf/uai/JinDAS17,DBLP:conf/nips/LevinePK11,DBLP:journals/corr/abs-1208-2112}.

\section{Statement of the Problem}
%clearly state the problem to be addressed in your forthcoming project. Explain
%why it would be worthwhile to solve this problem.

\begin{definition}
  A \emph{Markov decision process} (MDP) is a set $\mathcal{M} = \{ \mathcal{S},
  \mathcal{A}, \mathcal{T}, \gamma, r \}$, where $\mathcal{S}$ and
  $\mathcal{A}$ are sets of states and actions, respectively; $\mathcal{T} :
  \mathcal{S} \times \mathcal{A} \times \mathcal{S} \to [0, 1]$ is a function
  defined so that $\mathcal{T}(s, a, s')$ is the probability of moving to state $s'$
  after taking action $a$ in state $s$; $\gamma \in [0, 1)$ is the discount
  factor (with higher $\gamma$ values, it makes little difference whether a
  reward is received now or later, while with lower $\gamma$ values the future
  becomes gradually less and less important); and $r : \mathcal{S} \to
  \mathbb{R}$ is the reward function.
\end{definition}

In \emph{inverse reinforcement learning}, one is presented with an MDP without a
reward function $\mathcal{M} \setminus \{ r \}$ and a set of expert
demonstrations $\mathcal{D} = \{ \zeta_i \}_{i=1}^N$, where each demonstration
$\zeta_i = \{ (s_{i,0}, a_{i,0}), \dots, (s_{i,T}, a_{i,T}) \}$ is a multiset of
state-action pairs representing the actions taken by the expert during a
particular recorded session. Each state is also characterised by a number of
features. The goal of IRL is then to find $r$ such that the optimal policy under
$r$
\[ \pi^* = \argmax_\pi \mathbb{E}\left[ \sum_{t=0}^\infty \gamma^t r(s_t) | \pi
  \right] \]
matches the actions in $\mathcal{D}$.

The likelihood of the data can be written down as
\cite{DBLP:conf/uai/JinDAS17,DBLP:conf/nips/LevinePK11}
\begin{equation} \label{pDr}
  p(\mathcal{D} | r) = \prod_{i=1}^N \prod_{t=1}^T p(a_{i,t} | s_{i,t}) = \exp\left( \sum_{i=1}^N \sum_{t=1}^T Q(s_{i,t}, a_{i,t}; r) - V(s_{i,t}; r) \right),
\end{equation}
where
\[ Q(s_{i,t}, a_{i,t}; r) = r(s_{i,t}) + \gamma\sum_{s' \in \mathcal{S}}
  \mathcal{T}(s_{i,t}, a_{i,t}, s')V(s'; r), \]
and $V(s; r)$ can be obtained by repeatedly applying the equation
\cite{supplementary_material}
\[ V(s; r) = \log \sum_{a \in \mathcal{A}} \exp\left( r(s) + \gamma\sum_{s' \in
      \mathcal{S}} \mathcal{T}(s, a, s')V(s'; r) \right). \]
However, a reward function learned by maximising this likelihood is not
transferable to new situations
\cite{DBLP:conf/uai/JinDAS17,DBLP:conf/nips/LevinePK11}. One needs to model the
reward structure in a way that would allow reward predictions for previously
unseen states.

One way to model rewards without assumptions of linearity is with a
\emph{Gaussian process} (GP). A GP is a collection of random variables, any
finite combination of which has a joint Gaussian distribution
\cite{DBLP:books/lib/RasmussenW06}. We write $r \sim \mathcal{GP}(0,
k_{\Theta})$ to say that $r$ is a GP with mean $0$ and covariance function
$k_{\Theta}$, which uses a set of hyperparameters $\Theta$. Covariance functions
take two state feature vectors as input and quantify how similar the two states
are, in a sense that we would expect them to have similar rewards.

% TODO: import info from the MSci paper
As training a GP with $n$ data points has a time complexity of
$\mathcal{O}(n^3)$ \cite{DBLP:books/lib/RasmussenW06}, numerous approximation
methods have been suggested, many of which select a subset of data called
\emph{inducing points} and focus most of the training effort on them
\cite{DBLP:journals/corr/abs-1807-01065}. Let $\mathbf{X_u}$ be the matrix of
features at inducing states, $\mathbf{u}$ the rewards at those states, and
$\mathbf{r}$ a vector with $r(\mathcal{S})$ as elements. Then the full joint
probability distribution can be factorised as
\begin{equation} \label{full}
  \pfull = p(\mathbf{X_u}) \times p(\Theta | \mathbf{X_u}) \times p(\mathbf{u}
  | \Theta, \mathbf{X_u}) \times p(\mathbf{r} | \Theta, \mathbf{X_u},
  \mathbf{u}) \times p(\mathcal{D} | r).
\end{equation}
Here $p(\mathbf{X_u})$ and $p(\Theta | \mathbf{X_u})$ are freely chosen priors,
\[
  \begin{split}
    p(\mathbf{u} | \Theta, \mathbf{X_u}) &= \mathcal{N}(\mathbf{0},
    \Kuu) \\
    &= \frac{1}{(2\pi)^{n/2}|\Kuu|^{1/2}}\exp \left( -\frac{1}{2}
      \mathbf{u}^T\Kuu^{-1}\mathbf{u} \right) \\
    &= \exp\left(-\frac{1}{2}\mathbf{u}^T\Kuu^{-1}\mathbf{u} -
      \frac{1}{2}\log|\Kuu| - \frac{n}{2}\log 2\pi\right)
  \end{split}
\]
is the GP prior \cite{DBLP:books/lib/RasmussenW06}, the GP posterior is a
multivariate Gaussian \cite{DBLP:conf/nips/LevinePK11}
\begin{equation} \label{eq:r}
  p(\mathbf{r} | \Theta, \mathbf{X_u}, \mathbf{u}) =
  \mathcal{N}(\Kru^T\Kuu^{-1}\mathbf{u}, \Krr - \Kru^T\Kuu^{-1}\Kru),
\end{equation}
and $p(\mathcal{D} | r)$ is as in Equation \ref{pDr}. The matrices such as
$\Kru$ are called \emph{covariance matrices} and are defined as
$[\Kru]_{i,j} = k_\Theta(\mathbf{x}_{\mathbf{r},i}, \mathbf{x}_{\mathbf{u},j})$,
where $\mathbf{x}_{\mathbf{r},i}$ and $\mathbf{x}_{\mathbf{u},j}$ denote feature
vectors for the $i$th state in $\mathcal{S}$ and the $j$th state in
$\mathbf{X_u}$, respectively.

Given this model, data $\mathcal{D}$, and inducing feature matrix
$\mathbf{X_u}$, our goal is then to find optimal values of hyperparameters
$\Theta$, inducing rewards $\mathbf{u}$, and the reward function $r$. While the
previous paper that considered this IRL model computed maximum likelihood estimates
for $\Theta$ and $\mathbf{u}$, and made an assumption that $\mathbf{r}$ in
Equation \ref{eq:r} has $0$ variance \cite{DBLP:conf/nips/LevinePK11}, we aim to
avoid this assumption and use variational inference to approximate the full
posterior distribution $\posterior$. \emph{Variational inference} (VI) is an
approximation technique for probability densities \cite{blei2017variational}.
Let $\approximation$ be our approximating family of probability distributions
for $\posterior$ with its own set of hyperparameters $\Lambda$. Then it is up to
VI algorithms to optimise $\Lambda$ in order to minimise the
\emph{Kullback-Leibler} (KL) divergence between the original probability
distribution and our approximation.  KL divergence (asymmetrically) measures how
different the two distributions are and in this case can be defined as follows
\cite{blei2017variational}:
\[ \DKL(\approximation || \posterior) = \mathbb{E}_q[\log\approximation] -
  \mathbb{E}_q[\log \posterior]. \]
Since KL divergence is typically hard to compute, instead of minimising it, VI
typically tries to maximise the \emph{evidence lower bound} (ELBO) defined as
\[
  \begin{split}
    \mathcal{L}(\Lambda) &= \mathbb{E}_q[\log \pfull] -
    \mathbb{E}_q[\log\approximation] \\
    &= \iiint \approximation \log
    \frac{\pfull}{\approximation}\,d\Theta\,d\mathbf{u}\,d\mathbf{r}.
  \end{split}
\]
% TODO: cite something

By considering full probability distributions instead of point estimates,---as
long as the approximations are able to capture important features of the
posterior---our predictions are likely to be more accurate and rely on fewer
assumptions. Moreover, we hope to make use of various recent advancements in VI
for both time complexity and approximation distribution fit (see Section
\ref{literature}), making the resulting algorithm competitive both in terms of
speed and model fit.

\section{Literature Survey} \label{literature}

present an overview of relevant previous work including articles, books, and
existing software products. Critically evaluate the strengths and weaknesses of
the previous work.

\section{Proposed Approach}

%state how you propose to solve the software development problem. Show that your
%proposed approach is feasible, but identify any risks.

We begin by rewriting the posterior by using the chain rule and Bayes' theorem,
trying to extract parts of the distribution we know how to compute, namely those
in Equation \ref{full}:
\[
  \begin{split}
    \posterior &= p(\Theta | \mathbf{X_u}, \mathcal{D}) \times p(\mathbf{u} |
    \Theta, \mathbf{X_u}, \mathcal{D}) \times p(\mathbf{r} | \Theta,
    \mathbf{X_u}, \mathbf{u}, \mathcal{D}) \\
    &\propto p(\Theta | \mathbf{X_u}, \mathcal{D}) \times p(\mathbf{u} | \Theta,
    \mathbf{X_u}, \mathcal{D}) \times p(\mathcal{D} | r) \times p(\mathbf{r} |
    \Theta, \mathbf{X_u}, \mathbf{u}) \\
    &\propto p(\Theta | \mathbf{X_u}, \mathcal{D}) \times p(\mathcal{D} |
    \Theta, \mathbf{X_u}, \mathbf{u}) \times p(\mathbf{u} | \Theta,
    \mathbf{X_u}) \times p(\mathcal{D} | r) \times p(\mathbf{r} | \Theta,
    \mathbf{X_u}, \mathbf{u}) \\
    &\propto p(\mathcal{D} | \Theta, \mathbf{X_u}) \times p(\Theta |
    \mathbf{X_u}) \times p(\mathcal{D} | \Theta, \mathbf{X_u}, \mathbf{u})
    \times p(\mathbf{u} | \Theta, \mathbf{X_u}) \times p(\mathcal{D} | r) \times
    p(\mathbf{r} | \Theta, \mathbf{X_u}, \mathbf{u})
  \end{split}
\]
Note that now there are only two unknown probability distributions:
$p(\mathcal{D} | \Theta, \mathbf{X_u})$ and $p(\mathcal{D} | \Theta,
\mathbf{X_u}, \mathbf{u})$. They can be computed as follows:
\begin{align*}
  p(\mathcal{D} | \Theta, \mathbf{X_u}, \mathbf{u}) = \int &p(\mathcal{D} | r) \times p(\mathbf{r} | \Theta, \mathbf{X_u}, \mathbf{u}) \, d\mathbf{r}, \\
  p(\mathcal{D} | \Theta, \mathbf{X_u}) = \iint &p(\mathcal{D} | r) \times p(\mathbf{r} | \Theta, \mathbf{X_u}, \mathbf{u}) \times p(\mathbf{u} | \Theta, \mathbf{X_u}) \, d\mathbf{u} \, d\mathbf{r}.
\end{align*}

\section{Work Plan}

show how you plan to organize your work, identifying intermediate deliverables
and dates.

\section{Notes on papers (to be removed)}

\subsection{Miscellaneous}

(Directed) similarity between MDPs using restricted Boltzmann machines
\cite{9401f4eeb9a64c77afb3d087261d1080}

Chapter 6 on distance measures \cite{mccune2002analysis}

The PhD thesis behind maximum causal entropy \cite{Ziebart:2010:MPA:2049078}

\subsection{Gaussian Processes}

Simple introduction to GPs for time-series modelling \cite{Roberts2013GaussianPF}

Spectral kernels \cite{pmlr-v28-wilson13}

GPs over graphs instead of vectors (haven't actually read)
\cite{DBLP:journals/corr/abs-1803-05776}

Another introduction from physics (skimmed through) \cite{introduction_to_gps}

Learning a GP from very little data \cite{DBLP:conf/nips/PlattBSWZ01}

One GP for multiple correlated output variables \cite{DBLP:journals/jcphy/BilionisZKL13}

Kernels for categorical and count data \cite{savitsky2011variable}

Scalability/Approximations thesis \cite{kth}

\subsection{Interpretability}

Learning latent factors \cite{DBLP:conf/nips/LiSE17}

The behaviour of Reddit users \cite{DBLP:conf/atal/DasL14}

\subsection{Inverse Reinforcement Learning}

One of the first papers on the topic \cite{DBLP:conf/icml/NgR00}

Bayesian setting \cite{DBLP:conf/ijcai/RamachandranA07}

Learning optimal composite features \cite{DBLP:conf/ijcai/ChoiK13}

A different take on IRL with GPs \cite{DBLP:journals/corr/abs-1208-2112}

IRL for large state spaces (haven't read) \cite{DBLP:journals/jmlr/BoulariasKP11}

Multiple reward functions \cite{DBLP:conf/nips/ChoiK12}

A recent survey \cite{DBLP:journals/corr/abs-1806-06877}

Some not-very-successful method \cite{DBLP:conf/uai/NeuS07}

\subsubsection{Multiple Strategies}

EM clustering \cite{DBLP:conf/icml/BabesMLS11}

Structured priors \cite{DBLP:conf/ewrl/DimitrakakisR11}

There are more, but I haven't gotten to them yet.

\subsection{Variational Inference}

Part IV on probabilities and inference \cite{MacKay:2002:ITI:971143}

Normalizing flows \cite{DBLP:conf/icml/RezendeM15}

Linear VI for GPs \cite{DBLP:conf/nips/ChengB17}

Stochastic VI \cite{DBLP:journals/jmlr/HoffmanBWP13}

Structured stochastic VI (haven't read) \cite{DBLP:conf/aistats/HoffmanB15}

Another review of recent advances \cite{DBLP:journals/corr/abs-1711-05597}

Sparse VI for GP \cite{DBLP:journals/jmlr/HensmanDS17}

Sparse GPs \cite{DBLP:journals/jmlr/CandelaR05}.

Tighter ELBOs are not necessarily better \cite{DBLP:conf/icml/RainforthKLMIWT18}.

\cite{DBLP:books/lib/Bishop07}

\bibliographystyle{plain}
\bibliography{mprop}
\end{document}
