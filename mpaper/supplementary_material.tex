\documentclass{article}
\usepackage[UKenglish]{babel}
\usepackage[UKenglish]{isodate}
\usepackage{fullpage}
\usepackage{amsmath,amssymb,amsthm}
\usepackage{bm}
\usepackage{bbm}
\usepackage{mathtools}

\newtheorem{theorem}{Theorem}[section]
\newtheorem{proposition}[theorem]{Proposition}
\newtheorem{lemma}[theorem]{Lemma}
\newenvironment{proofsketch}{%
  \renewcommand{\proofname}{Proof sketch}\proof}{\endproof}
\theoremstyle{definition}
\newtheorem{definition}[theorem]{Definition}
\theoremstyle{remark}
\newtheorem*{remark}{Remark}

\DeclareMathOperator{\tr}{tr}

\newcommand{\Eq}{\mathbb{E}_{(\mathbf{u}, \mathbf{r}) \sim \approximation}}
\newcommand{\pfull}{p(\mathcal{D}, \mathbf{X_u}, \mathbf{u}, \mathbf{r})}
\newcommand{\approximation}{q_{\bm\nu}(\mathbf{u}, \mathbf{r})}
\newcommand{\Kuu}{\mathbf{K}_{\mathbf{u},\mathbf{u}}}
\newcommand{\Luu}{\mathbf{L}_{\mathbf{u},\mathbf{u}}}
\newcommand{\Krr}{\mathbf{K}_{\mathbf{r},\mathbf{r}}}
\newcommand{\Kru}{\mathbf{K}_{\mathbf{r},\mathbf{u}}}
\newcommand{\V}{V_{\mathbf{r}}}

\newcommand{\dm}{\frac{\partial}{\partial\bm\mu}}
\newcommand{\dS}{\frac{\partial}{\partial\bm\Sigma}}
\newcommand{\dt}{\frac{\partial}{\partial t}}
\newcommand{\dl}{\frac{\partial}{\partial \lambda_i}}
\newcommand{\dlj}{\frac{\partial}{\partial \lambda_j}}

\newcommand{\f}{f(\mathbf{r}, \mathbf{u}, t)}
\newcommand{\ftn}{f(\mathbf{r}, \mathbf{u}, t_n)}
\newcommand{\fn}{f_n(\mathbf{r}, \mathbf{u})}
\newcommand{\dx}{\,d\mathbf{r}\,d\mathbf{u}}
\newcommand{\df}{\left.\frac{\partial f}{\partial t}\right|_{(\mathbf{r},
    \mathbf{u}, t)}}
\newcommand{\g}{g(\mathbf{r}, \mathbf{u})}
\newcommand{\rinf}{\lVert \mathbf{r} \rVert_\infty}
\newcommand{\vbound}{\frac{\rinf + \log|\mathcal{A}|}{1 - \gamma}}

\title{Variational Inference for Inverse Reinforcement Learning with Gaussian
  Processes: Supplementary Material}
\author{Paulius Dilkas (2146879)}

\begin{document}
\maketitle

\section{Preliminaries}

For any matrix $\mathbf{A}$, we will use either $A_{i,j}$ or
$[\mathbf{A}]_{i,j}$ to denote the element of $\mathbf{A}$ in row $i$ and column
$j$.

For any vector $\mathbf{x}$, we write $\mathbb{R}_d[\mathbf{x}]$ to denote a
vector space of polynomials with degree at most $d$, where variables are
elements of $\mathbf{x}$, and coefficients are in $\mathbb{R}$.

In this paper, all references to measurability are with respect to the Lebesgue
measure. Similarly, whenever we consider the existence of an integral, we use
the Lebesgue definition of integration.

\begin{lemma}[Derivatives of probability
  distributions] \label{lemma:derivatives}
  \begin{enumerate}
    \leavevmode
  \item $\frac{\partial q(\mathbf{u})}{\partial \bm\mu} =
    q(\mathbf{u})\frac{1}{2}(\bm\Sigma^{-1} +
    \bm\Sigma^{-\intercal})(\mathbf{u} - \bm\mu)$.
  \item $\frac{\partial q(\mathbf{u})}{\partial \bm\Sigma} =
    -\frac{1}{2}\bm\Sigma^{-\intercal}q(\mathbf{u}) +
    \frac{1}{2}q(\mathbf{u})\bm\Sigma^{-\intercal}(\mathbf{u} -
    \bm\mu)(\mathbf{u} - \bm\mu)^\intercal\bm\Sigma^{-\intercal}$.
  \item For $i = 0, \dots, d$,
    \[
      \frac{\partial q(\mathbf{r})}{\partial \lambda_i} =
      q(\mathbf{r})\frac{1}{2}\tr
      \left((\Kuu^{-1}\mathbf{u}\mathbf{u}^\intercal\Kuu^{-\intercal} -
        \Kuu^{-1}) \frac{\partial \Kuu}{\partial \lambda_i} \right),
    \]
    where
    \[
      \frac{\partial \Kuu}{\partial \lambda_i} = \frac{1}{\lambda_i}\Kuu
    \]
    if $i = 0$, and
    \[
      \left[ \frac{\partial \Kuu}{\partial \lambda_i} \right]_{j,k} =
      k_{\bm\lambda}(\mathbf{x}_{\mathbf{u},j}, \mathbf{x}_{\mathbf{u},k})
      \left( -\frac{1}{2}(x_{\mathbf{u},j,i} - x_{\mathbf{u},k,i})^2 -
        \mathbbm{1}[j \ne k]\sigma^2 \right)
    \]
    otherwise.
  \end{enumerate}
\end{lemma}
\begin{proof}
  \leavevmode
  \begin{enumerate}
  \item
    \[
      \begin{split}
        \frac{\partial q(\mathbf{u})}{\partial m} &=
        q(\mathbf{u})\dm\left[-\frac{1}{2}(\mathbf{u} -
          \bm\mu)^\intercal\bm\Sigma^{-1}(\mathbf{u} - \bm\mu)\right]
        \\
        &= q(\mathbf{u})\left(-\frac{1}{2}\right)(\bm\Sigma^{-1} +
        \bm\Sigma^{-\intercal})(\mathbf{u} - \bm\mu)\dm[\mathbf{u} -
        \bm\mu] \\
        &= q(\mathbf{u})\frac{1}{2}(\bm\Sigma^{-1} +
        \bm\Sigma^{-\intercal})(\mathbf{u} - \bm\mu).
      \end{split}
    \]
  \item
    \[
      \begin{split}
        \frac{\partial q(\mathbf{u})}{\partial \bm\Sigma} &=
        \dS\left[\frac{1}{(2\pi)^{m/2}|\bm\Sigma|^{1/2}}\exp \left( -\frac{1}{2}
            (\mathbf{u} - \bm\mu)^\intercal\bm\Sigma^{-1}(\mathbf{u} -
            \bm\mu)\right)\right] \\
        &= \dS\left[\frac{1}{(2\pi)^{m/2}|\bm\Sigma|^{1/2}}\right]\exp \left( -\frac{1}{2}
          (\mathbf{u} - \bm\mu)^\intercal\bm\Sigma^{-1}(\mathbf{u} -
          \bm\mu)\right) \\
        &+ \frac{1}{(2\pi)^{m/2}|\bm\Sigma|^{1/2}}\dS\left[\exp\left( -\frac{1}{2}
            (\mathbf{u} - \bm\mu)^\intercal\bm\Sigma^{-1}(\mathbf{u} -
            \bm\mu)\right)\right] \\
        &=
        \frac{1}{(2\pi)^{m/2}}\dS\left[\frac{1}{|\bm\Sigma|^{1/2}}\right]\exp
        \left( -\frac{1}{2} (\mathbf{u} -
          \bm\mu)^\intercal\bm\Sigma^{-1}(\mathbf{u} - \bm\mu)\right) \\
        &- \frac{1}{2}q(\mathbf{u})\dS[(\mathbf{u}
        - \bm\mu)^\intercal\bm\Sigma^{-1}(\mathbf{u} - \bm\mu)]. \\
      \end{split}
    \]
    The two remaining derivatives can be taken with the help of \emph{The Matrix
      Cookbook} \cite{petersen2008matrix}:
    \begin{gather*}
      \dS\left[\frac{1}{|\bm\Sigma|^{1/2}}\right] =
      -\frac{1}{2}|\bm\Sigma|^{-3/2}\frac{\partial |\bm\Sigma|}{\partial \bm\Sigma} =
      -\frac{1}{2}|\bm\Sigma|^{-3/2}|\bm\Sigma|\bm\Sigma^{-\intercal} = -\frac{1}{2|\bm\Sigma|^{1/2}}\bm\Sigma^{-\intercal}, \\
      \dS[(\mathbf{u} - \bm\mu)^\intercal\bm\Sigma^{-1}(\mathbf{u} -
      \bm\mu)] = -\bm\Sigma^{-\intercal}(\mathbf{u} - \bm\mu)(\mathbf{u} -
                     \bm\mu)^\intercal\bm\Sigma^{-\intercal}.
    \end{gather*}
    Substituting them back in gives
    \[
      \frac{\partial q(\mathbf{u})}{\partial \bm\Sigma} =
      -\frac{1}{2}\bm\Sigma^{-\intercal}q(\mathbf{u}) +
      \frac{1}{2}q(\mathbf{u})\bm\Sigma^{-\intercal}(\mathbf{u} -
      \bm\mu)(\mathbf{u} - \bm\mu)^\intercal\bm\Sigma^{-\intercal}.
    \]
  \item Using a result by Rasmussen and Williams
    \cite{DBLP:books/lib/RasmussenW06},
    \[
      \frac{\partial q(\mathbf{r})}{\partial \lambda_i} = q(\mathbf{r}) \dl
      \left[-\frac{1}{2}\mathbf{u}^\intercal\Kuu^{-1}\mathbf{u} -
        \frac{1}{2}\log|\Kuu| \right] = q(\mathbf{r})\frac{1}{2}\tr
      \left((\Kuu^{-1}\mathbf{u}\mathbf{u}^\intercal\Kuu^{-\intercal} - \Kuu^{-1})
        \frac{\partial \Kuu}{\partial \lambda_i} \right).
    \]
    The remaining derivative is
    \[
      \frac{\partial \Kuu}{\partial \lambda_i} =
      \begin{cases}
        \frac{1}{\lambda_i}\Kuu & \text{if } i = 0, \\
        \Luu & \text{otherwise,}
      \end{cases}
    \]
    where
    \[
      \begin{split}
        [\Luu]_{j,k} &= \dl k_{\bm\lambda}(\mathbf{x}_{\mathbf{u},j},
        \mathbf{x}_{\mathbf{u},k}) \\
        &= k_{\bm\lambda}(\mathbf{x}_{\mathbf{u},j}, \mathbf{x}_{\mathbf{u},k})
        \dl \left[-\frac{1}{2}(\mathbf{x}_{\mathbf{u},j} -
          \mathbf{x}_{\mathbf{u},k})^\intercal \bm\Lambda
          (\mathbf{x}_{\mathbf{u},j} - \mathbf{x}_{\mathbf{u},k}) -
          \mathbbm{1}[j \ne k]\sigma^2\tr(\bm\Lambda) \right] \\
        &= k_{\bm\lambda}(\mathbf{x}_{\mathbf{u},j}, \mathbf{x}_{\mathbf{u},k})
        \dl \left[-\frac{1}{2}\sum_{l=1}^d \lambda_l
          (x_{\mathbf{u},j,l} - x_{\mathbf{u},k,l})^2 -
          \mathbbm{1}[j \ne k]\sigma^2\sum_{l=1}^d \lambda_l \right] \\
        &= k_{\bm\lambda}(\mathbf{x}_{\mathbf{u},j}, \mathbf{x}_{\mathbf{u},k})
        \left( -\frac{1}{2}(x_{\mathbf{u},j,i} -
        x_{\mathbf{u},k,i})^2 - \mathbbm{1}[j \ne k]\sigma^2 \right).
      \end{split}
    \]
  \end{enumerate}
\end{proof}

\subsection{Linear Algebra and Numerical Analysis}

\begin{definition}[Norms]
  For any finite-dimensional vector $\mathbf{x} = (x_1, \dots, x_n)^\intercal$,
  its \emph{maximum norm} is
  \[
    \lVert \mathbf{x} \rVert_\infty = \max_i |x_i|
  \]
  whereas its \emph{taxicab} (or \emph{Manhattan}) \emph{norm} is
  \[
    \lVert \mathbf{x} \rVert_1 = \sum_{i = 1}^n |x_i|.
  \]
  Let $\mathbf{A}$ be a matrix. For any vector norm $\lVert
  \cdot \rVert_p$, we can also define its \emph{induced norm} for matrices as
  \[
    \lVert \mathbf{A} \rVert_p = \sup_{\mathbf{x} \ne \mathbf{0}} \frac{\lVert
      \mathbf{Ax} \rVert_p}{\lVert \mathbf{x} \rVert_p}.
  \]
  In particular, for $p = \infty$, we have
  \[
    \lVert \mathbf{A} \rVert_\infty = \max_i \sum_{j} |A_{i,j}|.
  \]
\end{definition}

\begin{lemma}[Perturbation Lemma
  \cite{layton2014numerical}] \label{prop:condition_number}
  Let $\lVert \cdot \rVert$ be any matrix norm, and let $\mathbf{A}$ and
  $\mathbf{E}$ be matrices such that $\mathbf{A}$ is invertible and $\lVert
  \mathbf{A}^{-1} \rVert \lVert \mathbf{E} \rVert < 1$, then $\mathbf{A} +
  \mathbf{E}$ is invertible, and
  \[
    \lVert (\mathbf{A} + \mathbf{E})^{-1} \rVert \le \frac{\lVert
      \mathbf{A}^{-1} \rVert}{1 - \lVert \mathbf{A}^{-1} \rVert \lVert
      \mathbf{E} \rVert}.
  \]
\end{lemma}

\section{Proofs}

We primarily think of rewards as a vector $\mathbf{r} \in
\mathbb{R}^{|\mathcal{S}|}$, but sometimes we use a function notation $r(s)$ to
denote the reward of a particular state $s \in \mathcal{S}$. The functional
notation is purely a notational convenience.

MDP values are characterised by both a state and a reward function/vector. In
order to prove the next theorem, we think of the value function as $V :
\mathcal{S} \to \mathbb{R}^{|\mathcal{S}|} \to \mathbb{R}$, i.e., $V$ takes a
state $s \in \mathcal{S}$ and returns a function $V(s) :
\mathbb{R}^{\mathcal{S}} \to \mathbb{R}$ that takes a reward vector $\mathbf{r}
\in \mathbb{R}^{|\mathcal{S}|}$ and returns a value of the state $s$,
$\V(s) \in \mathbb{R}$. The function $V(s)$ computes the values of
all states and returns the value of state $s$.

\begin{proposition} \label{thm:measurability}
  MDP value functions $V(s) : \mathbb{R}^{|\mathcal{S}|} \to \mathbb{R}$ (for $s
  \in \mathcal{S}$) are Lebesgue measurable.
\end{proposition}
\begin{proofsketch}
  For any reward vector $\mathbf{r} \in \mathbb{R}^{|\mathcal{S}|}$, the
  collection of converged value functions $\{ \V(s) \mid s \in
  \mathcal{S} \}$ satisfy
  \[
    \V(s) = \log \sum_{a \in \mathcal{A}}
    \exp\left( r(s) + \gamma\sum_{s' \in \mathcal{S}} \mathcal{T}(s, a,
      s')\V(s') \right)
  \]
  for all $s \in \mathcal{S}$. Let $s_0 \in \mathcal{S}$ be an arbitrary state.
  In order to prove that $V(s_0)$ is measurable, it is enough to show that for
  any $\alpha \in \mathbb{R}$, the set
  \[
    \begin{split}
      \left\{ \vphantom{\sum_{a \in \mathcal{A}}} \right. \mathbf{r} \in
      \mathbb{R}^{|\mathcal{S}|} \left. \vphantom{\sum_{a \in \mathcal{A}}} \;
        \middle| \; \right. &\V(s_0) \in (-\infty, \alpha); \\
      &\V(s) \in
      \mathbb{R} \text{ for all } s \in \mathcal{S} \setminus \{ s_0 \}; \\
      &\V(s) = \left. \log \sum_{a \in \mathcal{A}} \exp\left( r(s)
          + \gamma\sum_{s' \in \mathcal{S}} \mathcal{T}(s, a,
          s')\V(s') \right) \text{ for all } s \in
        \mathcal{S}\right\}
    \end{split}
  \]
  is measurable. Since this set can be constructed in Zermelo-Fraenkel set
  theory \emph{without} the axiom of choice, it is measurable
  \cite{herrlich2006axiom}, which proves that $V(s)$ is a measurable function
  for any $s \in \mathcal{S}$.
\end{proofsketch}

\begin{proposition} \label{thm:bound}
  If the initial values of the MDP value function satisfy the following
  bound, then the bound remains satisfied throughout value iteration:
  \begin{equation} \label{eq:bound}
    |\V(s)| \le \vbound.
  \end{equation}
\end{proposition}
\begin{proof}
  We begin by considering \eqref{eq:bound} without taking the absolute value of
  $\V(s)$, i.e.,
  \begin{equation} \label{eq:positive_bound}
    \V(s) \le \vbound,
  \end{equation}
  and assuming that the initial values of $\{ \V(s) \mid s \in
  \mathcal{S} \}$ already satisfy \eqref{eq:positive_bound}. For each $s \in
  \mathcal{S}$, the value of $\V(s)$ is updated via this rule:
  \[
    \V(s) \coloneqq \log \sum_{a \in \mathcal{A}} \exp\left( r(s) +
      \gamma\sum_{s' \in \mathcal{S}} \mathcal{T}(s, a, s')\V(s')
    \right).
  \]
  Note that both $\log$ and $\exp$ are increasing functions, $\gamma > 0$, and
  the $\mathcal{T}$ function gives a probability (a non-negative number).
  Thus
  \[
    \begin{split}
      \V(s) &\le \log \sum_{a \in \mathcal{A}} \exp\left( r(s) + \gamma\sum_{s'
          \in \mathcal{S}} \mathcal{T}(s, a, s')\frac{\rinf +
          \log|\mathcal{A}|}{1 - \gamma} \right) \\
      &= \log \sum_{a \in \mathcal{A}} \exp\left( r(s) + \frac{\gamma (\rinf +
          \log|\mathcal{A}|)}{1 - \gamma}\sum_{s' \in \mathcal{S}}
        \mathcal{T}(s, a, s') \right) \\
      &= \log \sum_{a \in \mathcal{A}} \exp\left( r(s) + \frac{\gamma (\rinf +
          \log|\mathcal{A}|)}{1 - \gamma} \right)
    \end{split}
  \]
  by the definition of $\mathcal{T}$. Then
  \[
    \begin{split}
      \V(s) &\le \log \left( |\mathcal{A}| \exp\left( r(s) + \frac{\gamma
            (\rinf + \log|\mathcal{A}|)}{1 - \gamma} \right) \right) \\
      &= \log \left( \exp\left( \log|\mathcal{A}| + r(s) + \frac{\gamma
            (\rinf + \log|\mathcal{A}|)}{1 - \gamma} \right) \right) \\
      &= \log|\mathcal{A}| + r(s) + \frac{\gamma (\rinf + \log|\mathcal{A}|)}{1 - \gamma} \\
      &= \frac{\gamma (\rinf + \log|\mathcal{A}|) + (1 -
        \gamma)(\log|\mathcal{A}| + r(s))}{1 - \gamma} \\
      &\le \frac{\gamma (\rinf + \log|\mathcal{A}|) + (1 -
        \gamma)(\log|\mathcal{A}| + \rinf)}{1 - \gamma} \\
      &= \vbound
    \end{split}
  \]
  by the definition of $\rinf$.

  The proof for
  \begin{equation} \label{eq:negative_bound}
    \V(s) \ge \frac{\rinf + \log|\mathcal{A}|}{\gamma - 1}
  \end{equation}
  follows the same argument until we get to
  \[
    \begin{split}
      \V(s) &\ge \frac{\gamma(\rinf + \log|\mathcal{A}|) + (\gamma -
        1)(\log|\mathcal{A}| + r(s))}{\gamma - 1} \\
      &\ge \frac{\gamma(\rinf + \log|\mathcal{A}|) + (\gamma -
        1)(-\log|\mathcal{A}| -\rinf)}{\gamma - 1} \\
      &= \frac{\rinf + \log|\mathcal{A}|}{\gamma - 1},
    \end{split}
  \]
  where we use the fact that $r(s) \ge -\rinf - 2\log|\mathcal{A}|$. Combining
  \eqref{eq:positive_bound} and \eqref{eq:negative_bound} gives
  \eqref{eq:bound}.
\end{proof}

\begin{theorem}[The Lebesgue Dominated Convergence Theorem
  \cite{royden2010real}] \label{thm:lebesgue}
  Let $(X, \mathcal{M}, \mu)$ be a measure space and $\{ f_n \}$ a sequence of
  measurable functions on $X$ for which $\{ f_n \} \to f$ pointwise a.e. on $X$
  and the function $f$ is measurable. Assume there is a non-negative function
  $g$ that is integrable over $X$ and dominates the sequence $\{ f_n \}$ on $X$
  in the sense that
  \[
    |f_n| \le g \text{ a.e. on $X$ for all $n$.}
  \]
  Then $f$ is integrable over $X$ and
  \[
    \lim_{n \to \infty} \int_X f_n\,d\mu = \int_X f\,d\mu.
  \]
\end{theorem}

\begin{lemma} \label{lemma:bound1}
  Let $c : \mathbb{R}^{|\mathcal{S}|} \times \mathbb{R}^m \to (a, b) \subset
  \mathbb{R}$ be an arbitrary bounded function. Then, for $i = 0,
  \dots, d$,
  \[
    \left. \frac{\partial q(\mathbf{r})}{\partial \lambda_i} \right|_{\lambda_i
      = c(\mathbf{r}, \mathbf{u})}
  \]
  has upper and lower bounds of the form $q(\mathbf{r})d(\mathbf{u})$, where
  $d(\mathbf{u}) \in \mathbb{R}_2[\mathbf{u}]$.
\end{lemma}
\begin{proof}
  Remember that
  \[
    \frac{\partial q(\mathbf{r})}{\partial \lambda_i} =
    q(\mathbf{r})\frac{1}{2}\tr
    \left((\Kuu^{-1}\mathbf{u}\mathbf{u}^\intercal\Kuu^{-\intercal} - \Kuu^{-1})
      \frac{\partial \Kuu}{\partial \lambda_i}
    \right)
  \]
  by Lemma \ref{lemma:derivatives}. We begin by producing constant upper and
  lower bounds for the elements of
  \[
    \left. \frac{\partial \Kuu}{\partial \lambda_i} \right|_{\lambda_i =
      c(\mathbf{r}, \mathbf{u})}.
  \]
  If $i = 0$, then each element of $\frac{\partial
    \Kuu}{\partial \lambda_0}$ is of the form
  \[
    \exp \left( -\frac{1}{2}(\mathbf{x}_j - \mathbf{x}_k)^\intercal
      \bm\Lambda (\mathbf{x}_j - \mathbf{x}_k) - \mathbbm{1}[j \ne
      k]\sigma^2\tr(\bm\Lambda) \right),
  \]
  i.e., without $\lambda_0$, so
  \[
    \left. \frac{\partial \Kuu}{\partial \lambda_0} \right|_{\lambda_0 =
      c(\mathbf{r}, \mathbf{u})} = \frac{\partial \Kuu}{\partial \lambda_0}
  \]
  is already independent of $\mathbf{r}$ and $\mathbf{u}$---there is no need
  for any bounds.

  If $i > 0$, then each element of $\frac{\partial \Kuu}{\partial
    \lambda_i}$ is a constant multiple of $k_{\bm\lambda}(\mathbf{x}_j,
  \mathbf{x}_k)$, for some $\mathbf{x}_j$ and $\mathbf{x}_k$. Since
  $k_{\bm\lambda}(\mathbf{x}_j, \mathbf{x}_k)$ is a decreasing function of
  $\lambda_i$, and $c(\mathbf{r}, \mathbf{u}) > a$,
  \[
    \begin{split}
      k_{\bm\lambda}(\mathbf{x}_j, \mathbf{x}_k)|_{\lambda_i = 
        c(\mathbf{r}, \mathbf{u})} &=
      \begin{multlined}[t]
        \lambda_0 \exp \left( \vphantom{\sum_{n \in \{ \} \setminus}}
        \right. -\frac{1}{2}c(\mathbf{r}, \mathbf{u})(x_{j,i} - x_{k,i})^2 -
        \mathbbm{1}[j \ne k]\sigma^2c(\mathbf{r}, \mathbf{u}) \\
        - \left. \sum_{n \in \{ 1, \dots, d \} \setminus \{ i \}}
          \frac{1}{2} \lambda_n(x_{j,n} - x_{k,n})^2 + \mathbbm{1}[j \ne
          k]\sigma^2 \lambda_n \right)
      \end{multlined} \\
      &<
      \begin{multlined}[t]
        \lambda_0 \exp \left( \vphantom{\sum_{n \in \{ \} \setminus}}
        \right. -\frac{1}{2}a(x_{j,i} - x_{k,i})^2 -
        \mathbbm{1}[j \ne k]\sigma^2a \\
        - \left. \sum_{n \in \{ 1, \dots, d \} \setminus \{ i \}}
          \frac{1}{2} \lambda_n(x_{j,n} - x_{k,n})^2 + \mathbbm{1}[j \ne
          k]\sigma^2 \lambda_n \right),
      \end{multlined}
    \end{split}
  \]
  which gives an upper bound on each element of
  \[
    \left. \frac{\partial \Kuu}{\partial \lambda_i} \right|_{\lambda_i =
      c(\mathbf{r}, \mathbf{u})}.
  \]
  A similar line of reasoning establishes lower bounds as well.

  Combining the bounds with the observation that
  every element of
  $\Kuu^{-1}\mathbf{u}\mathbf{u}^\intercal\Kuu^{-\intercal}$ is in
  $\mathbb{R}_2[\mathbf{u}]$ gives the required result.
\end{proof}

\begin{remark}
  In order to find $\frac{\partial q(\mathbf{u})}{\partial t}$,
  where $t$ is the $i$th element of the vector $\bm\mu$, we can
  find $\frac{\partial q(\mathbf{u})}{\partial \bm\mu}$ and simply take the
  $i$th element. A similar line of reasoning applies to matrices as well. Thus,
  we only need to consider derivatives with respect to $\bm\mu$ and
  $\bm\Sigma$.
\end{remark}

\begin{lemma} \label{lemma:bound2}
  Let $c : \mathbb{R}^{|\mathcal{S}|} \times \mathbb{R}^m \to (a, b) \subset
  \mathbb{R}$ be an arbitrary bounded function. Then, for $i = 1, \dots, m$,
  every element of
  \[
    \left. \frac{\partial q(\mathbf{u})}{\partial \bm\mu} \right|_{\mu_i =
      c(\mathbf{r}, \mathbf{u})}
  \]
  has upper and lower bounds of the form $q(\mathbf{u})d(\mathbf{u})$,
  where $d(\mathbf{u}) \in \mathbb{R}_1[\mathbf{u}]$.
\end{lemma}
\begin{proof}
  Using Lemma \ref{lemma:derivatives},
  \[
    \left. \frac{\partial q(\mathbf{u})}{\partial \bm\mu} \right|_{\mu_i =
      c(\mathbf{r}, \mathbf{u})} = q(\mathbf{u})\frac{1}{2}(\bm\Sigma^{-1} +
    \bm\Sigma^{-\intercal})(\mathbf{u} - \mathbf{c}(\mathbf{r}, \mathbf{u})),
  \]
  where $\mathbf{c}(\mathbf{r}, \mathbf{u}) = (\mu_1, \dots, \mu_{i - 1},
  c(\mathbf{r}, \mathbf{u}), \mu_{i + 1} \dots, \mu_m)^\intercal$. Since
  $c(\mathbf{r}, \mathbf{u})$ is bounded and $\bm\Sigma^{-1} +
  \bm\Sigma^{-\intercal}$ is a constant matrix, we can use the bounds on
  $c(\mathbf{r}, \mathbf{u})$ to manufacture both upper and lower bounds on
  \[
     \left. \frac{\partial q(\mathbf{u})}{\partial \bm\mu} \right|_{\mu_i =
      c(\mathbf{r}, \mathbf{u})}
  \]
  of the required form.
\end{proof}

\begin{lemma} \label{lemma:bound3}
  Let $i, j = 1, \dots, m$, and let $\epsilon > 0$ be arbitrary. Furthermore,
  let
  \[
    c : \mathbb{R}^{|\mathcal{S}|} \times \mathbb{R}^m \to (\Sigma_{i,j} - \epsilon,
    \Sigma_{i,j} + \epsilon) \subset \mathbb{R}
  \]
  be a function with a codomain arbitrarily close to $\Sigma_{i,j}$. Then every
  element of
  \[
    \left. \frac{\partial q(\mathbf{u})}{\partial \bm\Sigma} \right|_{\Sigma_{i,j} =
    c(\mathbf{r}, \mathbf{u})}
  \]
  has upper and lower bounds of the form $q(\mathbf{u})d(\mathbf{u})$, where
  $d(\mathbf{u}) \in \mathbb{R}_2[\mathbf{u}]$.
\end{lemma}
\begin{proof}
  Using Lemma \ref{lemma:derivatives},
  \[
    \left. \frac{\partial q(\mathbf{u})}{\partial \bm\Sigma} \right|_{\Sigma_{i,j} =
    c(\mathbf{r}, \mathbf{u})} =
    -\frac{1}{2}\mathbf{C}(\mathbf{r}, \mathbf{u})^{-\intercal} +
    \frac{1}{2}\mathbf{C}(\mathbf{r}, \mathbf{u})^{-\intercal}(\mathbf{u} -
    \bm\mu)(\mathbf{u} -
    \bm\mu)^\intercal\mathbf{C}(\mathbf{r}, \mathbf{u})^{-\intercal},
  \]
  where
  \[
    [\mathbf{C}(\mathbf{r}, \mathbf{u})]_{k,l} =
    \begin{cases}
      c(\mathbf{r}, \mathbf{u}) & \text{if } (k, l) = (i, j), \\
      \Sigma_{k,l} & \text{otherwise.}
    \end{cases}
  \]
  We can also express $\mathbf{C}(\mathbf{r},\mathbf{u})$ as
  $\mathbf{C}(\mathbf{r}, \mathbf{u}) = \bm\Sigma + \mathbf{E}(\mathbf{r},
  \mathbf{u})$, where
  \[
    [\mathbf{E}(\mathbf{r}, \mathbf{u})]_{k,l} =
    \begin{cases}
      c(\mathbf{r}, \mathbf{u}) - \Sigma_{i,j} & \text{if } (k, l) = (i, j), \\
      0 & \text{otherwise.}
    \end{cases}
  \]
  We begin by establishing upper and lower bounds on $\mathbf{C}(\mathbf{r},
  \mathbf{u})^{-1}$. For this, we use the maximum norm $\lVert \cdot
  \rVert_\infty$ on both vectors and matrices. We can apply Lemma
  \ref{prop:condition_number} to $\bm\Sigma$ and $\mathbf{E}(\mathbf{r},
  \mathbf{u})$ since
  \[
    \lVert \mathbf{E}(\mathbf{r}, \mathbf{u}) \rVert_\infty = \max_k \sum_l
    |[\mathbf{E}(\mathbf{r}, \mathbf{u})]_{k,l}| = |c(\mathbf{r}, \mathbf{u}) -
    \Sigma_{i,j}| < \epsilon
  \]
  can be made arbitrarily small so that $\lVert \bm\Sigma^{-1} \rVert_\infty
  \lVert \mathbf{E}(\mathbf{r}, \mathbf{u}) \rVert_\infty < 1$. Then
  $\mathbf{C}(\mathbf{r}, \mathbf{u})$ is invertible, and
  \[
    \lVert \mathbf{C}(\mathbf{r}, \mathbf{u})^{-1} \rVert_\infty \le
    \frac{\lVert \bm\Sigma^{-1} \rVert_\infty}{1 - \lVert \bm\Sigma^{-1}
      \rVert_\infty \lVert \mathbf{E}(\mathbf{r}, \mathbf{u}) \rVert_\infty} <
    \frac{\lVert \bm\Sigma^{-1} \rVert_\infty}{1 - \lVert \bm\Sigma^{-1}
      \rVert_\infty \epsilon},
  \]
  which means that
  \[
    \max_k \sum_l \left| [\mathbf{C}(\mathbf{r}, \mathbf{u})^{-1}]_{k,l} \right|
    < \frac{\lVert \bm\Sigma^{-1} \rVert_\infty}{1 - \lVert \bm\Sigma^{-1}
      \rVert_\infty \epsilon},
  \]
  i.e., for any row $k$ and column $l$,
  \[
    \left| [\mathbf{C}(\mathbf{r}, \mathbf{u})^{-1}]_{k,l} \right| <
    \frac{\lVert \bm\Sigma^{-1} \rVert_\infty}{1 - \lVert \bm\Sigma^{-1}
      \rVert_\infty \epsilon},
  \]
  which bounds all elements of $\mathbf{C}(\mathbf{r}, \mathbf{u})^{-1}$ as
  required. Since every element of $(\mathbf{u} - \bm\mu)(\mathbf{u} -
  \bm\mu)^\intercal$ is in $\mathbb{R}_2[\mathbf{u}]$, and the elements of
  $\mathbf{C}(\mathbf{r}, \mathbf{u})^{-1}$ are bounded, the desired result
  follows.
\end{proof}

\begin{lemma} \label{lemma:integral_of_r}
  \begin{equation} \label{eq:r-inequality}
    \int \lVert \mathbf{r} \rVert_\infty q(\mathbf{r})\,d\mathbf{r} \le a +
    \lVert \Kru^\intercal \Kuu^{-1} \mathbf{u} \rVert_1,
  \end{equation}
  where $a$ is a constant independent of $\mathbf{u}$.
\end{lemma}
\begin{proof}
  Since $\rinf \le \lVert \mathbf{r} \rVert_1$,
  \[
    \int \lVert \mathbf{r} \rVert_\infty q(\mathbf{r})\,d\mathbf{r} \le \int
    \lVert \mathbf{r} \rVert_1 q(\mathbf{r})\,d\mathbf{r} =
    \sum_{i=1}^{|\mathcal{S}|} \mathbb{E}[|r_i|].
  \]
  As each $\mathbb{E}[|r_i|]$ is a mean of a folded Gaussian distribution,
  \[
    \mathbb{E}[|r_i|] = \sigma_i \sqrt{\frac{2}{\pi}} \exp
    \left(-\frac{\xi_i^2}{2\sigma_i^2} \right) + \xi_i \left( 1 - 2\Phi \left(
        -\frac{\xi_1}{\sigma_1} \right) \right),
  \]
  where $\xi_i = \left[\Kru^\intercal\Kuu^{-1}\mathbf{u}\right]_i$, $\sigma_i =
  \sqrt{[\Krr - \Kru^\intercal\Kuu^{-1}\Kru]_{i,i}}$\footnote{The expression
    under the square root sign is non-negative because $\Krr -
    \Kru^\intercal\Kuu^{-1}\Kru$ is a covariance matrix of a Gaussian
    distribution, hence also positive semi-definite, which means that its
    diagonal entries are non-negative.}, and $\Phi$ is the cumulative
  distribution function of the standard normal distribution. Furthermore,
  \[
    \mathbb{E}[|r_i|] \le \sigma_i\sqrt{\frac{2}{\pi}} + |\xi_i|,
  \]
  as $\sigma_i$ is non-negative, and $\Phi(x) \in [0, 1]$ for all $x$. Since
  \[ \sum_{i=1}^{|\mathcal{S}|} |\xi_i| = \lVert \Kru^\intercal \Kuu^{-1}
    \mathbf{u} \rVert_1, \]
  we can set
  \[ a = \sum_{i=1}^{|\mathcal{S}|} \sigma_i \sqrt{\frac{2}{\pi}} \]
  to get \eqref{eq:r-inequality}.
\end{proof}

Our main theorem is a specialised version of an integral differentiation result
by Chen \cite{lecture_notes}.
\begin{theorem} \label{thm:main}
  Whenever the derivative exists,
  \[
    \dt\iint
    \V(s)q(\mathbf{r})q(\mathbf{u})\dx
    = \iint
    \dt[\V(s)q(\mathbf{r})q(\mathbf{u})]\dx,
  \]
  where $t$ is any scalar part of $\bm\mu$, $\bm\Sigma$, or $\bm\lambda$.
\end{theorem}
\begin{proof}
  Let
  \begin{align*}
    \f &= \V(s)q(\mathbf{r})q(\mathbf{u}), \\
    F(t) &= \iint \f\dx,
  \end{align*}
  and fix the value of $t$. Let $(t_n)_{n=1}^\infty$ be any sequence such that
  $\lim_{n \to \infty} t_n = t$, but $t_n \ne t$ for all $n$. We want to show
  that
  \begin{equation} \label{eq:to_prove}
    F'(t) = \lim_{n \to \infty} \frac{F(t_n) - F(t)}{t_n - t} = \iint \df\dx.
  \end{equation}
  We have
  \[
    \frac{F(t_n) - F(t)}{t_n - t} = \iint \frac{\ftn - \f}{t_n - t}\dx =
    \iint \fn\dx,
  \]
  where
  \[
    \fn = \frac{\ftn - \f}{t_n - t}.
  \]
  Since
  \[
    \lim_{n \to \infty} \fn = \df,
  \]
  \eqref{eq:to_prove} follows from Theorem \ref{thm:lebesgue} as soon as we show
  that both $f$ and $f_n$ are measurable and find a non-negative integrable
  function $g$ such that for all $n$, $\mathbf{r}$, $\mathbf{u}$,
  \[
    |\fn| \le \g.
  \]
  The MDP value function is measurable by Proposition \ref{thm:measurability}.
  The result of multiplying or adding measurable functions (e.g., probability
  density functions (PDFs)) to a measurable function is still measurable. Thus,
  both $f$ and $f_n$ are measurable.

  It remains to find $g$. For notational simplicity and without loss of
  generality, we will temporarily assume that $t$ is a parameter of
  $q(\mathbf{r})$. Then
  \[
    |\fn| = |\V(s)| \left| \frac{q(\mathbf{r})|_{t =
          t_n} - q(\mathbf{r})}{t_n - t} \right| q(\mathbf{u})
  \]
  since PDFs are non-negative. An upper bound for
  $|\V(s)|$ is given by Proposition \ref{thm:bound}, while
  \[
    \frac{q(\mathbf{r})|_{t = t_n} - q(\mathbf{r})}{t_n - t} = \left.
      \frac{\partial q(\mathbf{r})}{\partial t} \right|_{t = c(\mathbf{r},
      \mathbf{u})}
  \]
  for some function $c : \mathbb{R}^{|\mathcal{S}|} \times \mathbb{R}^m \to
  (\min\{t, t_n\}, \max\{t, t_n\})$ due to the mean value theorem (since $q$ is
  a continuous and differentiable function of $t$, regardless of the specific
  choices of $q$ and $t$).

  We then have that
  \[
    |\fn| \le \vbound \left| \left. \frac{\partial
          q(\mathbf{r})}{\partial t} \right|_{t=c(\mathbf{r}, \mathbf{u})}
    \right| q(\mathbf{u}).
  \]
  The bound is clearly non-negative and measurable. It remains to show that it
  is also integrable. Depending on what $t$ represents, we can use one of the
  Lemmas \ref{lemma:bound1}, \ref{lemma:bound2}, and \ref{lemma:bound3}, which
  gives us two polynomials $p_1(\mathbf{u}), p_2(\mathbf{u}) \in
  \mathbb{R}_2[\mathbf{u}]$ such that
  \[
    p_1(\mathbf{u})q(\mathbf{r}) < \left. \frac{\partial q(\mathbf{r})}{\partial
        t} \right|_{t=c(\mathbf{r}, \mathbf{u})} < p_2(\mathbf{u})q(\mathbf{r}).
  \]
  Then
  \[
    \left| \left. \frac{\partial q(\mathbf{r})}{\partial t}
      \right|_{t=c(\mathbf{r}, \mathbf{u})} \right| < q(\mathbf{r}) \max \{
    |p_1(\mathbf{u})|, |p_2(\mathbf{u})| \}.
  \]
  We can now apply Lemma \ref{lemma:integral_of_r}, which allows us to integrate
  out $\mathbf{r}$, and we are left with showing the existence of
  \begin{equation} \label{eq:remaining_integral}
    \int \left( a + \lVert \Kru^\intercal \Kuu^{-1} \mathbf{u} \rVert_1 \right) \max \{|p_1(\mathbf{u})|, |p_2(\mathbf{u})| \} q(\mathbf{u})\,d\mathbf{u},
  \end{equation}
  where $a$ is a constant. The integral
  \[
    \int \max \{|p_1(\mathbf{u})|, |p_2(\mathbf{u})| \}
    q(\mathbf{u})\,d\mathbf{u} = \int \max \{|p_1(\mathbf{u})q(\mathbf{u})|,
    |p_2(\mathbf{u})q(\mathbf{u})| \}\,d\mathbf{u}
  \]
  exists because $p_1(\mathbf{u})q(\mathbf{u})$ and
  $p_2(\mathbf{u})q(\mathbf{u})$ are both integrable, hence their absolute
  values are integrable, and the maximum of two integrable functions is also
  integrable. Since $\lVert \Kru^\intercal \Kuu^{-1} \mathbf{u} \rVert_1 \in
  \mathbb{R}_1[\mathbf{u}]$, a similar argument can be applied to the rest of
  \eqref{eq:remaining_integral} as well.
\end{proof}

\section{Evidence Lower Bound}

\begin{equation} \label{eq:elbo}
  \begin{split}
    \mathcal{L} &= \Eq \left[ \log \frac{\pfull}{\approximation}
    \right] \\
    &= \iint \approximation \log
    \frac{\pfull}{\approximation}\dx.
  \end{split}
\end{equation}

\begin{equation} \label{eq:full}
  \pfull = p(\mathbf{X_u}) \times p(\mathbf{u} | \mathbf{X_u}) \times p(\mathbf{r} | \mathbf{X_u}, \mathbf{u}) \times p(\mathcal{D} | \mathbf{r}).
\end{equation}

\begin{equation} \label{eq:approximation}
  \approximation = q(\mathbf{u}) \times q(\mathbf{r} | \mathbf{u}).
\end{equation}

In this section we derive and simplify the ELBO for this (now fully specified)
model. In order to derive the ELBO, let us go back to \eqref{eq:elbo} and
write\footnote{At this point, we will drop the subscript denoting which
  variables the expectation is taken over. Also note that throughout the
  derivation equality is taken to mean `equality up to an additive constant'.}
\[
  \mathcal{L} = \mathbb{E}[\log\pfull] - \mathbb{E}[\log\approximation].
\]
By substituting in \eqref{eq:full} and \eqref{eq:approximation}, we get
\[
  \mathcal{L} = \mathbb{E}[\log p(\mathbf{X_u}) + \log p(\mathbf{u} |
  \mathbf{X_u}) + \log p(\mathbf{r} | \mathbf{X_u}, \mathbf{u}) + \log
  p(\mathcal{D} | \mathbf{r})] - \mathbb{E}[\log q(\mathbf{u}) + \log
  q(\mathbf{r} | \mathbf{u})].
\]
Note that $\mathbb{E}[\log p(\mathbf{X_u})]$ is just a constant, so we can
simply drop it from the expression. Furthermore, since $q(\mathbf{r} |
\mathbf{u}) = p(\mathbf{r} | \mathbf{X_u}, \mathbf{u})$, they cancel each other
out. Then, we can substitute various terms with their definitions to get
\[
  \mathcal{L} = \mathbb{E}[\log \mathcal{N}(\mathbf{u}; \mathbf{0}, \Kuu)] +
  \mathbb{E}\left[ \sum_{i=1}^N \sum_{t=1}^T Q_{\mathbf{r}}(s_{i,t}, a_{i,t}) -
    \V(s_{i,t}) \right] - \mathbb{E}[\log\mathcal{N}(\mathbf{u}; \bm\mu,
  \bm\Sigma)].
\]
Using the expressions for $Q_{\mathbf{r}}$ and the entropy of a normal
distribution \cite{DBLP:journals/tit/AhmedG89},
\[
  \mathcal{L} = \frac{1}{2}\log|\bm\Sigma| - \frac{1}{2}\log|\Kuu| +
  \mathbb{E}\left[\sum_{i=1}^N \sum_{t=1}^T r(s_{i,t}) - \V(s_{i,t}) +
    \gamma\sum_{s' \in \mathcal{S}} \mathcal{T}(s_{i,t}, a_{i,t}, s')\V(s')
  \right].
\]
We can simplify $\sum_{i=1}^N\sum_{t=1}^Tr(s_{i,t})$ by defining a new vector
$\mathbf{t} = (t_1, \dots, t_{|\mathcal{S}|})^\intercal$, where $t_i$ is the
number of times the state associated with the reward $r_i$ has been visited
across all demonstrations. Then
\[
  \mathbb{E} \left[ \sum_{i=1}^N\sum_{t=1}^Tr(s_{i,t}) \right] =
  \mathbb{E}[\mathbf{t}^\intercal\mathbf{r}] =
  \mathbf{t}^\intercal\mathbb{E}[\mathbf{r}] =
  \mathbf{t}^\intercal\mathbb{E}\left[\Kru^\intercal\Kuu^{-1}\mathbf{u}\right] =
  \mathbf{t}^\intercal\Kru^\intercal\Kuu^{-1}\bm\mu.
\]
This allows us to simplify $\mathcal{L}$ to
\[
  \mathcal{L} = \frac{1}{2}\log|\bm\Sigma| - \frac{1}{2}\log|\Kuu| +
  \mathbf{t}^\intercal\Kru^\intercal\Kuu^{-1}\bm\mu - \sum_{i=1}^N \sum_{t=1}^T
  \mathbb{E}[\V(s_{i,t})] - \gamma\sum_{s' \in \mathcal{S}}
  \mathcal{T}(s_{i,t}, a_{i,t}, s')\mathbb{E}[\V(s')].
\]

% TODO: move this somewhere else
% TODO: source (wikipedia)
We use Cholesky decomposition $\bm\Sigma = \mathbf{L}\mathbf{L}^\intercal$,
where $\mathbf{L} \in \mathbb{R}^{m \times m}$ is a lower triangular matrix with
positive diagonal entries. As this decomposition is bijective as a mapping
between positive-definite real matrices and $\mathbf{L}$ as described
previously, we can construct any viable covariance matrix $\bm\Sigma$ from
$\mathbf{L}$ (except matrices that are positive semi-definite but not positive
definite, which is a reasonable compromise).

\subsection{$\partial/\partial\bm\mu$}

We begin by removing terms independent of $\bm\mu$:
\[
  \frac{\partial\mathcal{L}}{\partial\bm\mu} =
  \dm[\mathbf{t}^\intercal\Kru^\intercal\Kuu^{-1}\bm\mu] -
  \sum_{i=1}^N \sum_{t=1}^T \dm\mathbb{E}[\V(s_{i,t})] -
  \gamma\sum_{s' \in \mathcal{S}} \mathcal{T}(s_{i,t}, a_{i,t},
  s')\dm\mathbb{E}[\V(s')].
\]
Here
\[
  \dm\mathbb{E}[\V(s)] = \dm\iint \V(s) q(\mathbf{r})
  q(\mathbf{u})\dx = \iint \V(s) q(\mathbf{r}) \frac{\partial
    q(\mathbf{u})}{\partial \bm\mu}\dx = \frac{1}{2}\mathbb{E}[\V(s)
  (\bm\Sigma^{-1} + \bm\Sigma^{-\intercal})(\mathbf{u} - \bm\mu)]
\]
by Theorem \ref{thm:main} and Lemma \ref{lemma:derivatives}.
Hence
\[
  \frac{\partial\mathcal{L}}{\partial\bm\mu} =
  \mathbf{t}^\intercal\Kru^\intercal\Kuu^{-1} -
  \frac{1}{2}\sum_{i=1}^N\sum_{t=1}^T \mathbb{E}[\V(s_{i,t})
  (\bm\Sigma^{-1} + \bm\Sigma^{-\intercal})(\mathbf{u} - \bm\mu)] -
  \gamma\sum_{s' \in \mathcal{S}} \mathcal{T}(s_{i,t}, a_{i,t}, s')
  \mathbb{E}[\V(s') (\bm\Sigma^{-1} +
  \bm\Sigma^{-\intercal})(\mathbf{u} - \bm\mu)].
\]

\subsection{$\partial/\partial\bm\Sigma$}
% TODO: d/dL

Similarly to the previous section,
\[
  \frac{\partial\mathcal{L}}{\partial\bm\Sigma} = \frac{1}{2}\dS\log|\bm\Sigma|
  - \sum_{i=1}^N \sum_{t=1}^T \dS\mathbb{E}[\V(s_{i,t})] -
  \gamma\sum_{s' \in \mathcal{S}} \mathcal{T}(s_{i,t}, a_{i,t},
  s')\dS\mathbb{E}[\V(s')],
\]
where $\dS\log|\bm\Sigma| = \bm\Sigma^{-\intercal}$ by Petersen and Pedersen
\cite{petersen2008matrix}, and
\[
  \dS\mathbb{E}[\V(s)] = \iint \V(s) q(\mathbf{r})
  \frac{\partial q(\mathbf{u})}{\partial \bm\Sigma}\dx =
  \frac{1}{2}\mathbb{E}[\V(s)(\bm\Sigma^{-\intercal}(\mathbf{u} -
  \bm\mu)(\mathbf{u} - \bm\mu)^\intercal\bm\Sigma^{-\intercal} -
  \bm\Sigma^{-\intercal})],
\]
by Theorem \ref{thm:main} and Lemma \ref{lemma:derivatives}. Therefore,
\[
  \begin{split}
    \frac{\partial\mathcal{L}}{\partial\bm\Sigma} &=
    \frac{1}{2}\bm\Sigma^{-\intercal} - \frac{1}{2} \sum_{i=1}^N\sum_{t=1}^T
    \mathbb{E}[\V(s_{i,t})(\bm\Sigma^{-\intercal}(\mathbf{u} -
    \bm\mu)(\mathbf{u} - \bm\mu)^\intercal\bm\Sigma^{-\intercal} -
    \bm\Sigma^{-\intercal})] \\
    &- \gamma\sum_{s' \in \mathcal{S}}\mathcal{T}(s_{i,t}, a_{i,t}, s')
    \mathbb{E}[\V(s')(\bm\Sigma^{-\intercal}(\mathbf{u} -
    \bm\mu)(\mathbf{u} - \bm\mu)^\intercal\bm\Sigma^{-\intercal} -
    \bm\Sigma^{-\intercal})].
  \end{split}
\]

\subsection{$\partial/\partial \lambda_j$}

For $j = 0, \dots, d$,
\[
  \frac{\partial \mathcal{L}}{\partial \lambda_j} = - \frac{1}{2}\dlj\log|\Kuu|
  + \mathbf{t}^\intercal\dlj \left[ \Kru^\intercal\Kuu^{-1} \right] \bm\mu -
  \sum_{i=1}^N \sum_{t=1}^T \dlj\mathbb{E}[\V(s_{i,t})] -
  \gamma\sum_{s' \in \mathcal{S}} \mathcal{T}(s_{i,t}, a_{i,t},
  s')\dlj\mathbb{E}[\V(s')],
\]
where
\begin{align*}
  \dlj\log|\Kuu| &= \tr \left( \Kuu^{-1} \frac{\partial \Kuu}{\partial \lambda_j}
  \right), \\
  \dlj \left[ \Kru^\intercal\Kuu^{-1} \right] &= \frac{\partial
    \Kru^\intercal}{\partial \lambda_j} \Kuu^{-1} + \Kru^\intercal
  \frac{\partial \Kuu^{-1}}{\partial \lambda_j} = \frac{\partial
    \Kru^\intercal}{\partial \lambda_j} \Kuu^{-1} -
  \Kru^\intercal\Kuu^{-1}\frac{\partial \Kuu}{\partial \lambda_j}\Kuu^{-1}
\end{align*}
by Petersen and Pedersen \cite{petersen2008matrix}, and
\[
  \dlj \mathbb{E}[\V(s)] = \iint\V(s)\frac{\partial q(\mathbf{r})}{\partial
    \lambda_j}q(\mathbf{u})\dx = \frac{1}{2}\mathbb{E} \left[ \V(s) \tr
    \left((\Kuu^{-1}\mathbf{u}\mathbf{u}^\intercal\Kuu^{-\intercal} - \Kuu^{-1})
      \frac{\partial \Kuu}{\partial \lambda_i} \right) \right]
\]
by Theorem \ref{thm:main} and Lemma \ref{lemma:derivatives}. Thus,
\[
  \begin{split}
    \frac{\partial \mathcal{L}}{\partial \lambda_j} = &- \frac{1}{2} \tr
    \left(\Kuu^{-1} \frac{\partial \Kuu}{\partial \lambda_j} \right) +
    \mathbf{t}^\intercal \left( \frac{\partial \Kru^\intercal}{\partial
        \lambda_j} - \Kru^\intercal\Kuu^{-1}\frac{\partial
        \Kuu}{\partial \lambda_j} \right) \Kuu^{-1} \bm\mu \\
    &- \frac{1}{2} \sum_{i=1}^N \sum_{t=1}^T \mathbb{E} \left[ \V(s_{i,t}) \tr
      \left((\Kuu^{-1}\mathbf{u}\mathbf{u}^\intercal\Kuu^{-\intercal} -
        \Kuu^{-1}) \frac{\partial \Kuu}{\partial \lambda_i} \right) \right] \\
    &- \gamma\sum_{s' \in \mathcal{S}} \mathcal{T}(s_{i,t}, a_{i,t}, s')
    \mathbb{E} \left[ \V(s') \tr
      \left((\Kuu^{-1}\mathbf{u}\mathbf{u}^\intercal\Kuu^{-\intercal} -
        \Kuu^{-1}) \frac{\partial \Kuu}{\partial \lambda_i} \right) \right],
  \end{split}
\]
where the remaining derivatives can be found in Lemma \ref{lemma:derivatives}.

\bibliographystyle{abbrv}
\bibliography{paper.bib}
\end{document}